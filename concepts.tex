\begin{frame}{Enrichissement de la liste des concepts}
	\begin{center}
		Termes \underline{inventés} par Charcot :
	\end{center}
	\begin{table}[h]
			\small
	\begin{tabular}{rl}
	
		nom traditionnel & nom moderne / synonyme \\ 		\hline
		\bolder{paralysie agitante} & maladie de Parkinson\\
		\bolder{ataxie locomotrice progressive} & \textit{tabes dorsalis}\\
		\bolder{arthropathies tabétiques} & arthropathie de Charcot\\
%		\bolder{trépidation épileptoïde du pied}  & clonus\\
		\bolder{sclérose latérale amyotrophique} & maladie de Charcot / Lou Gehrig\\
		\bolder{idée(s) fixe(s)}, \bolder{maladie des tics} &  syndrome de Tourette\\
		$\dots$
%		\bolder{mouvements involontaires}  & 	chorées, athétose\\
%		\bolder{incapacité d'être debout / de marcher} & astasie-abasie\\
%		\bolder{atrophie musculaire progressive} & maladie Charcot-Marie-Tooth
	\end{tabular}
\end{table}
	\medskip
	\begin{center}
			$\neq$ Termes \underline{transmis} par Charcot :
	\end{center}
	
	
	
		\small
		\begin{tabular}{rl}
			\textcolor{deepblue}{\textbf{athétose}} &  mouvements involontaires \\
			 \textcolor{deepblue}{\textbf{hystérie}} &  névrose \\ \textcolor{deepblue}{\textbf{épilepsie}} &  attaques convulsives \\ \textcolor{deepblue}{\textbf{hypnose}}  &  transe \\ \textcolor{deepblue}{\textbf{sclérose en plaques disséminées}} &
			  sclérose multiple
		\end{tabular}
		
			\begin{flushright}
			\scriptsize
			(\citealp{walusinski,camargo2023})
		\end{flushright}

	
	
\end{frame}
