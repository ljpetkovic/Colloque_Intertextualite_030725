% !TeX program = xetex
\documentclass[xcolor={table,usenames,dvipsnames}]{beamer}
\usepackage{pgfpages}
%\usepackage{pdfcomment}
%\newcommand{\pdfnote}[1]{\marginnote{\pdfcomment[icon=note]{#1}}}



%\setbeameroption{hide notes} % Only slides
%\setbeameroption{show only notes} % Only notes
%\setbeameroption{show notes on second screen=right} % Both
\setbeamertemplate{note page}{\pagecolor{yellow!5}\insertnote}\usepackage{palatino}
\usepackage{eso-pic} 
\usepackage[absolute,overlay]{textpos}
\usepackage{colortbl}
\usepackage{fourier}
\usepackage{booktabs}% http://ctan.org/pkg/booktabs
\newcommand{\tabitem}{~~\llap{\textbullet}~~}
\usepackage{tabularx}
\usepackage{tikz}
\usetikzlibrary{positioning}
\usetikzlibrary{arrows.meta}

\setbeamertemplate{blocks}[rounded][shadow=true]
\let\olditem\item
\renewcommand{\item}{%
\olditem\vspace{0pt}}     
\usepackage{ragged2e}


%\usepackage[round]{natbib} % incompatible avec biblatex
\usepackage{hyperref}
\hypersetup{
    colorlinks=true,
    linkcolor=.,
    filecolor=deepblue,      
    urlcolor=deepblue,
    pdftitle={Overleaf Example},
    pdfpagemode=FullScreen,
    citecolor=deepblue
    }
\definecolor{LightCyan}{rgb}{0.88,1,1}   
\usepackage[justification=centering]{caption}
\captionsetup{font=scriptsize}
\captionsetup[figure]{name=Fig.}
\captionsetup[table]{name=Tab.}
\setbeamertemplate{caption}[numbered]
\usepackage[T1]{fontenc}
\usepackage{ctex}
\UseRawInputEncoding
%\usepackage[backend=bibtex, style=authoryear, natbib=true, sorting=nty, backref=true]{biblatex}
%%%% TOUCHE PLUS LE STYLE DE CITATION !!
\usepackage[style=authoryear, maxbibnames=99, mincitenames=1, maxcitenames=2, backref=true, hyperref=true, dashed=false, firstinits=true, backend=bibtex, bibencoding=utf8, uniquename=false, uniquelist=false, natbib=true]{biblatex}
\renewcommand*{\bibfont}{\footnotesize}
\setbeamerfont{footnote}{size=\tiny}

% Remove quotation marks from titles
\DeclareFieldFormat[article,incollection,inproceedings,conference]{title}{#1} 
\addbibresource{bibliographie.bib}

%\usepackage[backend=bibtex,
%style=authoryear,
%natbib=true,
%sorting=nty,
%backref=true
%]{biblatex}

\let\oldnocite\nocite
\makeatletter
\renewcommand*{\nocite}[1]{\oldnocite{#1}\Hy@backout{#1}}
\makeatother

\renewcommand*{\bibfont}{\footnotesize}

\DeclareCiteCommand{\cite}
  {\usebibmacro{prenote}}
  {\usebibmacro{citeindex}%
   \printtext[bibhyperref]{\usebibmacro{cite}}}
  {\multicitedelim}
  {\usebibmacro{postnote}}

\DeclareCiteCommand*{\cite}
  {\usebibmacro{prenote}}
  {\usebibmacro{citeindex}%
   \printtext[bibhyperref]{\usebibmacro{citeyear}}}
  {\multicitedelim}
  {\usebibmacro{postnote}}

\DeclareCiteCommand{\parencite}[\mkbibparens]
  {\usebibmacro{prenote}}
  {\usebibmacro{citeindex}%
    \printtext[bibhyperref]{\usebibmacro{cite}}}
  {\multicitedelim}
  {\usebibmacro{postnote}}

\DeclareCiteCommand*{\parencite}[\mkbibparens]
  {\usebibmacro{prenote}}
  {\usebibmacro{citeindex}%
    \printtext[bibhyperref]{\usebibmacro{citeyear}}}
  {\multicitedelim}
  {\usebibmacro{postnote}}

\DeclareCiteCommand{\footcite}[\mkbibfootnote]
  {\usebibmacro{prenote}}
  {\usebibmacro{citeindex}%
  \printtext[bibhyperref]{ \usebibmacro{cite}}}
  {\multicitedelim}
  {\usebibmacro{postnote}}

\DeclareCiteCommand{\footcitetext}[\mkbibfootnotetext]
  {\usebibmacro{prenote}}
  {\usebibmacro{citeindex}%
   \printtext[bibhyperref]{\usebibmacro{cite}}}
  {\multicitedelim}
  {\usebibmacro{postnote}}

%\DeclareCiteCommand{\textcite}
%  {\boolfalse{cbx:parens}}
%  {\usebibmacro{citeindex}%
%   \printtext[bibhyperref]{\usebibmacro{textcite}}}
%  {\ifbool{cbx:parens}
%     {\bibcloseparen\global\boolfalse{cbx:parens}}
%     {}%
%   \multicitedelim}
%  {\usebibmacro{textcite:postnote}}

        \DeclareCiteCommand{\textcite}
        {\usebibmacro{cite:init}%
            \usebibmacro{prenote}}
        {\usebibmacro{citeindex}%
            \printtext[bibhyperref]{\usebibmacro{textcite}}}
        {}
        {\printtext[bibhyperref]{\usebibmacro{textcite:postnote}}%
            \usebibmacro{cite:post}}

%\addbibresource{bibliographie.bib}

% Cannot enable in Xelatex
\usepackage{pgfpages}
% \setbeameroption{hide notes} % Only slides
% \setbeameroption{show only notes} % Only notes
% \setbeameroption{show notes on second screen}

% other packages
\usepackage{latexsym,amsmath,multicol,booktabs,calligra}
\usepackage{graphicx,listings,stackengine}
\usepackage[greek,french]{babel}
\usepackage[LGR,T1]{fontenc}
\usepackage{fontspec}

%\usepackage[sfdefault,light,led=.85]{merriweather} %% Option 'black' gives heavier bold face 


\usepackage[sfdefault]{AlegreyaSans} %% Option 'black' gives heavier bold face
%% The 'sfdefault' option to make the base font sans serif
\renewcommand*\oldstylenums[1]{{\AlegreyaSansOsF #1}}



% Define a command for text in Greek. Replace 'Gentium Plus' with a font of your choice if necessary.
%\newfontfamily\greekfont{Gentium Plus}
%\newcommand{\textgreek}[1]{{\greekfont #1}}

\DefineBibliographyStrings{french}{%
  backrefpage = {voir p\adddot},%
  backrefpages = {voir pp\adddot}%
}
\DeclareFieldFormat{pagerefformat}{\mkbibparens{{\color{red}\mkbibemph{#1}}}}
\renewbibmacro*{pageref}{%
  \iflistundef{pageref}
    {}
    {\printtext[pagerefformat]{%
       \ifnumgreater{\value{pageref}}{1}
         {\bibstring{backrefpages}\ppspace}
         {\bibstring{backrefpage}\ppspace}%
       \printlist[pageref][-\value{listtotal}]{pageref}}}}
\usepackage{wasysym}
% Enable only in Xelatex
 \usepackage{pstricks}

\author[Ljudmila PETKOVIC]{\small \textbf{Ljudmila PETKOVIC}\textsuperscript{1,2,3,4}\\\medskip{\footnotesize\texttt{prenom.nom@sorbonne-universite.fr}}}
\title[Extraction terminologique $\cdot$ circulation des savoirs $\cdot$ corpus Charcot]{\fontsize{13pt}{16pt}\selectfont Mesurer l'impact de Jean-Martin Charcot sur la littérature médicale en français dans la période \textsc{XIX}\ieme{}-\textsc{XX}\ieme{} siècle}
%\subtitle{Approche \textit{PatternRank}}
\institute [JE \og{}Humanités numériques\fg{}] {\tiny \textsuperscript{1} Sorbonne Université, Faculté des Lettres, \textsc{UFR} Littératures françaises et comparée, \textsc{ED III} (\textsc{ED019})\\\textsuperscript{2} Sorbonne Université, Centre d'étude de la langue et des littératures françaises (\textsc{CELLF}), \textsc{UMR 8599}\\\textsuperscript{3} Sorbonne Université, Observatoire des textes, des idées et des corpus (\textsc{ObTIC})\\\textsuperscript{4} Sorbonne Université, \textsc{UFR} Sociologie et Informatique pour les Sciences Humaines}
\date[Colloque Intertextualité, 03/07/2025]{\scriptsize Colloque \og{}Le texte de l'autre. Dialogue interdisciplinaire $\dots$\fg{} \\École des Chartes, salle Léopold-Delisle\\Paris, le 3 juillet 2025}
\usepackage{YTU}

% defs
\def\cmd#1{\texttt{\color{red}\footnotesize $\backslash$#1}}
\def\env#1{\texttt{\color{blue}\footnotesize #1}}
\definecolor{deepblue}{rgb}{0,0,0.5}
\definecolor{deepred}{rgb}{0.6,0,0}
\definecolor{deepgreen}{rgb}{0,0.5,0}
\definecolor{halfgray}{gray}{0.55}
\definecolor{warmblack}{rgb}{0.0, 0.26, 0.26}
\newcommand{\bolder}[1]{{\color{purple}\bfseries#1}}
\lstset{
    basicstyle=\ttfamily\small,
    keywordstyle=\bfseries\color{deepblue},
    emphstyle=\ttfamily\color{deepred},    % Custom highlighting style
    stringstyle=\color{deepgreen},
    numbers=left,
    numberstyle=\small\color{halfgray},
    rulesepcolor=\color{red!20!green!20!blue!20},
    frame=shadowbox,
}
% \logo{%
%     \includegraphics[width=1cm,height=1cm,keepaspectratio]{pic/obtic.jpg}~%
%     \includegraphics[width=1cm,height=1cm,keepaspectratio]{pic/Lettres_su_logo.png}~%
% }
\usepackage{enumerate}
%\setbeamertemplate{section in toc}{\hspace*{1em}\inserttocsectionnumber.~\inserttocsection\par}
\setbeamertemplate{subsection in toc}{\hspace*{2em}\inserttocsectionnumber.\inserttocsubsectionnumber.~\inserttocsubsection\par}
\renewcommand*{\bibfont}{\scriptsize}



\let\oldfootnotesize\footnotesize
\renewcommand*{\footnotesize}{\oldfootnotesize\scriptsize}

%\setbeamertemplate{itemize/enumerate body begin}{\small}
\setbeamertemplate{itemize/enumerate subbody begin}{\small}
%
%\newcommand{\leftquote}{{\fontfamily{lmr}\selectfont\textquotedblleft}}
%\newcommand{\rightquote}{{\fontfamily{lmr}\selectfont\textquotedblright}}
%\newcommand{\leftguillemet}{{\fontfamily{lmr}\selectfont\guillemotleft}}
%\newcommand{\rightguillemet}{{\fontfamily{lmr}\selectfont\guillemotright}}


\setbeamertemplate{itemize subitem}{\textcolor{blue}{$\circ$}}




\begin{document}



\begin{frame}
    \titlepage
%     \note[item]{Welcome to the talk!}
\note{Bonjour à tout$\cdot$e$\cdot$s, je vous remercie pour votre accueil.\\\bigskip}

\note{Je vais vous présenter mon projet de thèse visant à mesurer l'impact du neurologue J.-M. Charcot sur la littérature médicale en français dans la période \textsc{XIX}\ieme{}-\textsc{XX}\ieme{} s.\\\bigskip}


\note{J'ai légèrement modifié l'intitulé de ma présentation pour une raison particulière que je vais détailler tout à l'heure.}


\begin{figure}
    \centering
    
    \includegraphics[width=1cm,height=1cm,keepaspectratio]{pic/Lettres_su_logo.png}~\hspace*{0.3cm}%\includegraphics{}
    \includegraphics[width=1cm,height=1cm,keepaspectratio]{pic/cellf.png}~\hspace*{0.5cm}%
    \includegraphics[width=2cm,height=1cm,keepaspectratio]{pic/obtic.jpg}~%


\end{figure}
    
%    \begin{note}
%        {Introduce your self}
%    \end{note}

\end{frame}



\begin{frame}{Sommaire}
	\tableofcontents[subsectionstyle=show/show/hide]
\end{frame}

\section[Projet Charcot]{Projet Charcot}
\subsection[Présentation générale du sujet]{Présentation générale du sujet}
\begin{frame}{Entre l'histoire des sciences et des humanités numériques}
	 \note{Cette thèse, intitulée... est guidée par trois axes structurants, notamment la valorisation... la circulation des savoirs (et non du discours du point de vue narratif) et l'intertextualité.\\}
	 \note{Elle est dirigée par Glenn Roe et co-encadrée par Motasem Alrahabi, et est née de l'initiative de l'Observatoire des Patrimoines de Sorbonne Université (OPUS) qui a financé le projet.}
	\vspace{-3ex}
\begin{table}[h]
	\centering
	\begin{tabular}{|c|}
		\hline
		\fontsize{9}{10}\selectfont \textit{Dans les petits papiers de Charcot : de l'expérimentation aux prémisses de la neurologie moderne\footnote{\url{https://theses.fr/s382733}}} \\
		\hline
		\rowcolor{yellow!30} \multicolumn{1}{c}{\small Valorisation numérique des archives de Jean-Martin Charcot} \\
		\rowcolor{blue!10} \multicolumn{1}{c}{\small Circulation des savoirs $\cdot$ intertextualité} \\
	\end{tabular}
\end{table}

	\begin{columns}
		\column{0.40\textwidth}
		Thèse en cours (2021--) \\
		\footnotesize{initiative \textsc{OPUS}}%
		\setcounter{footnote}{1}% Force the counter to "b"
		\footnote{\url{https://institut-opus.sorbonne-universite.fr/node/478}}\\
		\begin{itemize}
			\footnotesize
			\item dir. : Prof. D\textsuperscript{r} Glenn ROE
			\item co-enc. : D\textsuperscript{r} Motasem ALRAHABI
		\end{itemize}
		\column{0.50\textwidth}
		\begin{figure}
			\centering
			\includegraphics[width=.20\textwidth]{pic/Jean-Martin_Charcot-modified.png}
			\caption{J.-M. Charcot (1825-1893), \href{https://fr.wikipedia.org/wiki/Jean-Martin_Charcot\#/media/Fichier:Jean-Martin\_Charcot.jpg}\\
				{Wikipédia}.}
		\end{figure}
					\begin{itemize}
			\small
			\item père de la neurologie moderne
			\item contributions et influences : 
			\begin{itemize}
				\footnotesize
				\item hystérie, \og{}Parkinson\fg{}, \textsc{SLA}$\dots$
				\item Freud, de la Tourette, Babinski$\dots$
			\end{itemize}
			\item héritage scientifique vivant

			
			%			\item Fonds Charcot sur SorbonNum\footnote{\url{https://patrimoine.sorbonne-universite.fr}}
		\end{itemize}
	\end{columns}

\end{frame}

\subsection[Contexte et importance de l'étude]{Contexte et importance de l'étude}
\input{contexte_importance}
\subsection[Objectifs de la recherche]{Objectifs de la recherche}
\input{objectif_double}
\section[Problématique et hypothèses]{Problématique et hypothèses}
\subsection[Question de recherche]{Question de recherche}
\begin{frame}{Problématique principale}
	%\justifying 
	
\begin{table}[h]
	\note{rupture épistémologique de Bachelard}
	\begin{tabular}{ll}
		diffusion des constructions symboliques & théories, concepts \\
		produites par leur concepteur & Charcot\\
		repérées dans les ouvrages de son réseau & Autres \\
		influence +/- & succès, innovation, controverse \\
	\end{tabular}
\end{table}
%		\begin{flushright}
%	\small
%	\citep[pp.~221-222]{quet2014frederic}
%\end{flushright}


%Circulation de savoirs : 
%\begin{columns}
%	contenu...
%\end{columns}
%\begin{itemize}
%	\item diffusion des constructions symboliques \hfill théories, concepts
%	\item produites par leur concepteur \hfill Charcot
%	\item repérées dans les ouvrages de son réseau
%	\item influence +/- \hfill succès, innovation, controverse
%\end{itemize}
	\begin{block}{Évaluer l’influence de Charcot \textit{via} les termes repris dans son réseau}
					\centering
					\textcolor{black}{Quels concepts médicaux associés à Charcot ont eu un impact computationnellement mesurable sur son réseau scientifique ?}
%	\textcolor{blue}{*}
				\end{block}

\bigskip
%Concepts scientifiques en \textsc{TAL} :
%\begin{itemize}
%	\item termes candidats ou entités nommées (\textsc{EN}) propres à un domaine ;
%	\item détectables par des patrons linguistiques (morpho-syntaxiques) ;
%	\item ancrage référentiel : valeur sémantique particulière à leur contexte ;
%	\item pertinence élevée selon les métriques de pondération des termes.
%\end{itemize}		
%\begin{flushright}
%\small
%\citep[pp.~1-2]{omrane2011poids}: 
%\end{flushright}		
				
\end{frame}
\subsection[Hypothèses formulées]{Hypothèses formulées}
\input{hypotheses}
\section{Extraction de la terminologie (\textit{ATE})}
\subsection[Méthodes de collecte de donnée]{Méthodes de collecte de données}
\input{corpus}
\subsection[\textit{Design} de la recherche]{\textit{Design} de la recherche}
\input{methodo}
\input{concepts}
\subsection[Outils et techniques utilisées]{Outils et techniques utilisées}
\begin{frame}{Approches comparées}
%	Méthodes classiques \textit{vs.} celles de l'état de l'art.
	\begin{enumerate}
		\item \textcolor{deepblue}{\textbf{\texttt{TermSuite}}} \hfill {\small\citep{cram2016terminology}}
%		 \footnote{\url{https://github.com/ljpetkovic/Charcot_TermSuite}} \citep{cram2016terminology}
		\begin{itemize}
			\item linguistique (règles) + \textsc{TF-IDF} 
		\end{itemize} 
		\item \textcolor{deepblue}{\textbf{TF-IDF, BM25}} \hfill {\small\citep{robertson1976relevance}}
%		\footnote{\url{https://github.com/ljpetkovic/Charcot_circulations}} \citep{robertson1976relevance}  
		\begin{itemize}
			\item statistique, pondération des termes
		\end{itemize}
		\item \textcolor{deepblue}{\textbf{\textit{PatternRank}}} \hfill {\small \citep{schopf2022}}
%		\footnote{\url{https://github.com/ljpetkovic/Seminaire_doctoral_ObTIC_130325/blob/main/0_main.pdf}}
		\begin{itemize}
			\item apprentissage profond
			\item \texttt{keybert} + \texttt{keyphrase-vectorizers}
			\item utilisation des étiquettes POS
		\end{itemize} 
	\end{enumerate}
	
	\begin{block}{\vspace*{-0.6mm}}
		Traitements effectués en local (1,2) et \textit{via} la plateforme MeSU\footnote{\url{https://sacado.sorbonne-universite.fr/fr/plateforme-mesu/}} (3).
		\begin{itemize}
			\item appliqués à tout le corpus
		\end{itemize}
	\end{block}
	
\end{frame}
\input{non_supervise}
\section{Résultats}
\subsection[Présentation des résultats principaux]{Présentation des résultats principaux}









%\begin{frame}{Limitations de \texttt{keybert}}
%	\danger{} manque de diversification des résultats + (non-)grammaticalité\\
%	\begin{figure}[!ht]
%		\centering
%		\includegraphics[width=110mm,scale=0.5]{pic/termes\_keybert\_autres.png}
%		\caption{Répartition des 15 termes les plus pertinents dans le corpus \og{}Autres\fg{} selon \texttt{keybert}.}
%		\label{fig:enter-label}
%	\end{figure}
%\end{frame}

%\begin{frame}{Phrases-clés \textit{hapax} partagés dans les deux corpus selon \texttt{keybert}}
%	Les seuls termes partagés avec le corpus Charcot : 
%	%\begin{itemize}
%	%\item articulations de [\textit{sic}] épaule
%	%\item paralysie faciale périphérique
%	%\end{itemize}
%	\begin{figure}[!ht]
%		\centering
%		\includegraphics[width=90mm,scale=0.5]{pic/termes\_partages\_keybert.png}
%		\caption{Répartition des termes les plus pertinents dans les deux corpus selon \texttt{keybert}.}
%		\label{fig:enter-label}
%	\end{figure}
%\end{frame}


%\begin{frame}{Termes partagés extraits avec \texttt{keyphrase-vectorizers}}
%    \begin{figure}[!ht]
	%        \centering
	%        \includegraphics[width=85mm,scale=0.5]{pic/visualisation_termes_dupliques.png}
	%        \caption{Les termes communs aux deux corpus selon \texttt{keyphrase-vectorizers}.}
	%        \label{fig:enter-label}
	%    \end{figure}
%\end{frame}

%\begin{frame}{Termes partagés | \texttt{keyphrase-vectorizers}}
%    \begin{figure}[!ht]
	%        \centering
	%        \includegraphics[width=100mm,scale=0.5]{pic/termes_partages_liens.png}
	%        \caption{Les termes communs (fréq. = 1) aux deux corpus selon \texttt{keyphrase-vectorizers}.}
	%        \label{fig:enter-label}
	%    \end{figure}
%\end{frame}




\begin{frame}{Les termes partagés les plus fréquents | \texttt{keyphrase-vectorizers}}
	\begin{figure}[!ht]
		\centering
		\includegraphics[width=100mm,scale=0.5]{pic/termes_partages.png}
		\caption{Les 15 termes les plus fréquents dans les deux corpus selon \texttt{keyphrase-vectorizers}.}
		\label{fig:enter-label}
	\end{figure}
\end{frame}



\begin{frame}{Les termes les plus impactants}
	\begin{itemize}
		\item  \textbf{tics convulsifs} (\textit{PatternRank}), \textbf{hypnose} (moyenne)
		\item \textit{PatternRank} valorise systématiquement les termes
		\item pas de consensus entre les métriques
%		\begin{itemize}
%			\item l'écart le plus petit entre eux : \textit{hypnose}
%		\end{itemize}
	\end{itemize}
	\begin{figure}[h]
		\centering
		\includegraphics[width=\linewidth]{pic/termes_viz.png}
		\caption{Scores de pertinences pour chaque terme de référence, corpus \og{}Autres\fg{}.}
		\label{fig:ling_out_TAL}
	\end{figure}
\end{frame}



%\begin{frame}{Accès à la plateforme technologique \textsc{MeSU}}
%\begin{itemize}
%\item expériences réalisées sur la plateforme \href{https://sacado.sorbonne-universite.fr/}{\textsc{MeSU}} de Sorbonne Université
%\end{itemize}
%\bigskip
%
%Les données et les scripts utilisés dans le cadre de cette étude sont disponibles sur le \href{https://github.com/ljpetkovic/JE\_IA\_HN\_030524}{dépôt GitHub}.
%\end{frame}

%\begin{frame}{Analyse comparative des approches employées}
%	\begin{itemize}
%		\item \textit{PatternRank} valorise systématiquement les termes
%		\item pas de consensus entre les métriques
%		\begin{itemize}
%			\item l'écart le plus petit entre eux : \textit{hypnose}
%		\end{itemize}
%	\end{itemize}
%	\begin{figure}[h]
%		\includegraphics[width=\linewidth]{pic/termes_viz.png}
%		\caption{Visualisation des scores de pertinences pour chaque terme de référence.}
%		\label{fig:ling_out_TAL}
%	\end{figure}
%\end{frame}
\subsection[Analyse et interprétation des résultats]{Analyse et interprétation des résultats}
\begin{frame}{Analyse des concordances des termes médicaux}
	Analyses effectuées dans \textsc{TXM}.
\begin{itemize}
	\item On utilise un terme dans le contexte où on cite Charcot :
\begin{itemize}
	\item Terme traditionnel (ex. \textit{paralysie agitante})
	\item Synonyme ou terme relevant du champ conceptuel :
	\begin{itemize}
		\item \textit{pied tabétique} $\rightarrow$ arthropathies tabétiques
	\end{itemize}
\end{itemize}
\end{itemize}

	\begin{block}{Exemple}
		\texttt{([word = "paralysie"] [word = "agitante"] []* [word = "Charcot"] | [word = "Charcot"] []* [word = "paralysie"] [word = "agitante"]) within p}
		\begin{itemize}
			\item repérer toutes les occurrences dans un paragraphe où \textit{paralysie agitante} et \textit{Charcot} apparaissent dans n'importe quel ordre, séparés par 0 ou plusieurs mots
		\end{itemize}
	\end{block}
\end{frame}


\begin{frame}{Références à Charcot}
		\begin{table}
		\centering
		\resizebox{\textwidth}{!}{%
			\begin{tabular}{|l|p{10cm}|}
				\hline
				\textbf{Terme} & \textbf{Contexte} \\
				\hline
				\textit{épilepsie} & M. \textbf{Charcot} a décrit avec le plus grand soin l'\underline{épilepsie} partielle d'origine syphilitique [$\dots$] \\
				\hline
				\textit{hypnose} & Les trois états de l'\underline{hypnose} décrits par M. \textbf{Charcot} sont devenus classiques, [$\dots$] \\
				\hline
				\textit{localisations cérébrales} & Je vous ai montré \textbf{Charcot}, concourant pour
				la plus grosse part, à l'édification de la doctrine des \underline{localisations cérébrales}, qui est devenue quelque chose comme la préface d'une psychologie nouvelle.\\
				\hline
				\textit{embarras parole} & Lorsqu'on se trouve en présence d'un malade ayant de l'\underline{embarras de la parole} [$\dots$] A. La réponse à la première proposition n'est nullement embarrassante, si l'on veut se rappeler ces paroles de M. le professeur \textbf{Charcot} : [$\dots$] \\
				\hline
				\textit{tics convulsifs} & désignée par M. \textbf{Charcot} sous le nom de maladie des \underline{tics convulsifs} \\
				\hline
			\end{tabular}
		}
		\caption{Concordance des termes médicaux faisant référence à Charcot -- corpus \og{}Autres\fg{}.}
	\end{table}
\end{frame}


\begin{frame}{Analyse des cooccurrences}
	\begin{itemize}
		\item quels cooccurrents avec les termes médicaux ciblés ? 
		\begin{itemize}
			\item Charcot, Babinski, Necker$\dots$
		\end{itemize}
		\item recensement des résultats pour le cooccurrent : Charcot
		\item sinon, autre cooccurrent (médecin) avec l'indice le plus élevé
		\begin{itemize}
			\item termes créés par d'autres médecins, mais popularisés par Charcot
		\end{itemize}
	
		\begin{alertblock}{Exemple}
			\texttt{[word = "athétose"]} 
			\begin{itemize}
				\item liste des cooccurrents pour le terme \textit{athétose}
			\end{itemize}
		\end{alertblock}
	\end{itemize}
\end{frame}

\begin{frame}{Exemples de cooccurrences}
	
	Termes associés avec d'autres médecins : Sydenham, Jackson, Chervin$\dots$ 
		\begin{figure}[h]
		\centering
		\includegraphics[width=\linewidth]{pic/cooccurrences.png}
		\caption{Indice de cooccurrence par terme (Charcot \textit{vs.} Autres.}
		\label{fig:ling_out_TAL}
	\end{figure}
\end{frame}


%\begin{frame}{Analyse des cooccurrences}
%	Absence occasionnelle de cooccurrent Charcot expliquable :
%	\begin{itemize}
%		\item \textit{épilepsie} : terme créé par J. H. Jackson
%		\item \textit{hypnose} : terme créé par J. Braid
%		\item \textit{athétose} : terme créé par W. A. Hammond
%		\item \textit{chorées} : définition moderne par T. Sydenham
%		\item \textit{trépidation épileptoïde du pied} : Babinski ? Vulpian? Charcot ? 			
%	\end{itemize}
%\end{frame}

%\begin{frame}{Analyse des autres cooccurrents}
%	\begin{table}
%		\centering
%		\resizebox{\textwidth}{!}{%
%			\begin{tabular}{|l|p{10cm}|}
%				\hline
%				\textbf{Terme} & \textbf{Contexte} \\
%				\hline
%				\textit{trépidation épileptoïde du pied} (\textit{clonus}) & \textbf{On} [Babinski] le désigne alors sous la dénomination de \og{}\underline{clonus du
%					pied}\fg{}, \og{}\underline{trépidation épileptoïde du pied}\fg{} \\ \hline
%				\textit{tremblement} &  D'après \textbf{Charcot} et surtout d'après \textbf{Achard}, ce \underline{tremblement} aurait de certaines analogies avec le tremblement sénile, [$\dots$] \\ \hline
%				\textit{nystagmus} & [$\dots$] les recherches inspirées par les
%				travaux de \textbf{Barany} sur le \underline{nystagmus} provoqué indiqueraient une certaine fréquence de troubles labyrinthiques [$\dots$] \\ \hline
%				\textit{embarras parole} & M. le Dc \textbf{Chervin}, [$\dots$] vient de rédiger un nouveau résumé des notions cliniques fondamentales indispensables à connaître sur quelques \underline{troubles fonctionnels de la parole} et notamment sur le bégaiement. 
%				\\
%				\hline
%				\textit{athétose} & [$\dots$] nous avons affaire au symptôme désigné par M. W. \textbf{Hammond} sous le nom d'\underline{athétose}.\\
%				\hline
%				\textit{chorées} & Cependant, la \underline{chorée} de \textbf{Sydenham} présente quelques particularités sémiologiques que nous allons passer en revue.\\ \hline 
%				
%				épilepsie & Depuis les remarquables travaux de M. Hughlings \textbf{Jackson} sur la forme d'\underline{épilepsie} à laquelle il a attaché son nom, [$\dots$] \\ \hline
%				\textit{hypnose} & Selon \textbf{Braid}, l'\underline{hypnose} est caractérisée par des phénomènes mentaux et physiques, particuliers à cette condition.\\
%				\hline
%			\end{tabular}
%		}
%		\caption{Concordance des termes médicaux faisant référence à d'autres médecins -- corpus Autres.}
%	\end{table}
%\end{frame}

%\begin{frame}{Cas ambigu de la \textit{trépidation épileptoïde du pied} (\textit{clonus})}
%	
%	\begin{quote}
%		\textbf{On} le désigne alors sous la dénomination de \og{}\underline{clonus du pied}\fg{}, de \og{}\underline{trépidation épileptoïde du pied}\fg{}.
%	\end{quote}
%	\begin{flushright}
%		\small
%		\parencite{babinski1934oeuvre}
%	\end{flushright}
%	
%	
%	
%	\begin{quote}
%		\small
%		[$\dots$] l'un d'eux, [$\dots$], est
%		connu en France sous le nom de \underline{trépidation provoquée},	\underline{d'épilepsie spinale provoquée}. Les auteurs allemands rappelent le phénomène du pied ‭{\underline{Fussph\oe{}nomen}}, ou encore le \underline{clonus du pied}.
%		Mais c'est là un signe qui appartient ‭à la clinique française. Dès ‭1863, [$\dots$], il ‭était journellement mis ‭à profit dans les services de la Salpêtrière, par M. \textbf{Vulpian}, par \textbf{moi-même} et par \textbf{nos ‭élèves}.
%	\end{quote}
%	
%	\begin{flushright}
%		\small
%		\parencite{brissaud1893}
%	\end{flushright}
%	%		CL\_000001\_004 (pas inclus dans le corpus Autres -.-)
%	%		\bigskip
%	%					\begin{quote}
%		%			“$\dots$ known in France under the name of \underline{provoked trepidation}, or \underline{provoked spinal epilepsy}. German writers call it the foot-phenomenon (\underline{Fussphoenomen}) or \underline{ankle clonus}. But \textbf{the discovery of this sign belongs to French clinical observers}. Since 1863$\dots$ it has been \textbf{practised} daily in the wards of La Salpêtrière \textbf{by M. Vulpian, by myself} {\normalfont[Charcot]}, \textbf{and by our pupils}.”
%		%		\end{quote}
%	%		\begin{flushright}
%		%			\small
%		%			\citep{charcot1883lectures}
%		%		\end{flushright}
%	%	
%\end{frame}

%\begin{frame}{Analyse comparative des approches employées}
%	\begin{itemize}
	%		\item \textit{PatternRank} valorise systématiquement les termes
	%		\item pas de consensus entre les métriques
	%		\begin{itemize}
		%			\item l'écart le plus petit entre eux : \textit{hypnose}
		%		\end{itemize}
	%	\end{itemize}
%\begin{figure}[h]
%	\includegraphics[width=\linewidth]{pic/termes_viz.png}
%	\caption{Visualisation des scores de pertinences pour chaque terme de référence}
%	\label{fig:ling_out_TAL}
%\end{figure}
%\end{frame}

%\begin{frame}{\textit{PatternRank}}
%	Les termes les plus pertinents dans \og{}Autres\fg{} :
%\begin{table}[h]
%	\centering
%	\begin{tabular}{|l|l|c|}
	%		\hline
	%		\textbf{Terme} & \textbf{Synonyme} & \textbf{Score} \\
	%		\hline
	%		\texttt{tics convulsifs} & \bolder{syndrome de Tourette} & \textsc{0.8331} \\
	%		\texttt{état parkinsonien} & \bolder{maladie de Parkinson} & 0.7936 \\
	%%		\texttt{paralytiques agitants} & \bolder{maladie de Parkinson} & 0.7851 \\
	%		\hline
	%	\end{tabular}
%\end{table}
%\end{frame}





%\begin{frame}{Chronologie d'une locution : indice de croissance de l'impact ?}
%\begin{figure}[h] % Use [H] to force the figure to stay in place
%	\begin{itemize}
	%		\item évolution de la fréquence des termes au sein des deux corpus\footnote{\url{https://obtic.huma-num.fr/obvie/charcot/?view=corpus}}
	%		\item convergence entre des termes : fin \textsc{XIX}\ieme{}, début \textsc{XX}\ieme{} s.
	%%		\begin{itemize}
		%%				\item \textit{ppm} : nombre d’occurrences par million de mots 
		%%		\end{itemize}
	%	\end{itemize}
%	\centering
%	\includegraphics[width=\linewidth]{pic/tics_convulsifs.png}
%	\caption{Chronologie de la fréquence du terme \textit{tic convulsif}.}
%	\label{fig:ling_out_TAL}
%\end{figure}
%
%\begin{figure}[h]
%	\includegraphics[width=\linewidth]{pic/arthropathies_tabetiques.png}
%\caption{Chronologie de la fréquence du terme \textit{arthropathies tabétiques}.}
%\label{fig:ling_out_TAL}
%\end{figure}
%\end{frame}

\section[Conclusion]{Conclusion et perspectives}
\begin{frame}{Conclusion et perspectives}
	\begin{enumerate}
		\item 	\textit{PatternRank} : la méthode la plus robuste
		\begin{itemize}
			\item capture les termes composés (1-5-grammes)
			\begin{itemize}
				%				\item quadrigrammes : \textit{sclérose cérébrale tubéreuse hypertrophique} 
				\item \textit{méningite syphilitique hémorragique fibrineuse aiguë}
			\end{itemize}
			\item globalement, les scores plus élevés 
%			\begin{itemize}
%				\item exception : scores BM25 (\textit{SLA}, \textit{embarras parole}) et TF-IDF (\textit{hypnose})
%			\end{itemize}
		\end{itemize}
		
		\item 	les termes les plus impactants :
		\begin{itemize}
%			\item Charcot : \textit{hystérie}, \textit{astasie-abasie}, \textit{embarras parole}
			\item \textit{tics convulsifs}, \textit{épilepsie}, \textit{atrophie musculaire progressive} (\textit{PatternRank})
			\item \textit{hypnose}, \textit{embarras parole}, \textit{tics convulsifs} (globalement)
		\end{itemize}

		%		\item Absence des scores pour les termes comme \textit{SEP} et SLA :
		%		\begin{itemize}                                        
			%			\item solution : chercher leurs symptomes ou leurs descriptions :
			%			\begin{itemize}
				%				\item  \textit{amyotrophie spinale progressive}, \textit{secousses nystagmiques}$\dots$
				%			\end{itemize}
			%		\end{itemize}
	\end{enumerate}
	
	\begin{alertblock}{\vspace*{-0.6mm}}
		\centering
		Les résultats sont alignés avec les faits historiques.
	\end{alertblock}
	
	\bigskip
	{\color{RubineRed} \textbf{Recherches futures}}
	\begin{itemize}
		\item refaire les expériences, échantillon \og{}Autres\fg{} (1881-1934)
	\end{itemize}
%	tester les \textit{LLM} ou les \textit{LCM} (angl. \textit{Large Concept Models}) ?
\end{frame}

\begin{frame}{Sources}
	Dépôts GitHub :
	\begin{itemize}
		\small
		\item \url{https://github.com/ljpetkovic/Charcot_TermSuite}
		\item \url{https://github.com/ljpetkovic/Charcot_circulations}
		\item \url{https://github.com/ljpetkovic/Seminaire_doctoral_ObTIC_130325}
	\end{itemize}
\end{frame}


\begin{frame}[allowframebreaks]{Références}
\printbibliography

\end{frame}


\begin{frame}
\begin{center}
	\Large Annexe
\end{center}
\end{frame}

\begin{frame}{Domaine impactant (globalement) : \textbf{hypnose}}
	
	\begin{table}[h]
		\resizebox{\textwidth}{!}{%
			\centering
			\begin{tabular}{|l|r|r|r|r|r|}
				\hline
				\multicolumn{1}{|c|}{\textbf{Terme}} &
				\multicolumn{1}{c|}{\textbf{TF-IDF} (\textit{TermSuite})} & 	
				\multicolumn{1}{c|}{\textbf{TF-IDF}} & 	
				\multicolumn{1}{c|}{\textbf{BM25}} & 	
				\multicolumn{1}{c|}{\textit{\textbf{PatternRank}}} & 
				\multicolumn{1}{c|}{\textbf{Moyenne}} \\
				\hline
				\rowcolor{yellow!30}\textit{hypnose} & 0,3543 & 1 & 0,2922 & 0,7738 & \bolder{0,6050} \\
				\textit{embarras parole} & NA & 0,0018 & 0,9347 & NA & 0,4683 \\
				\textit{tics convulsifs} & NA & 0,1293 & 0,8385 & 0,8331 & 0,4502 \\
				\textit{arthropathies tabétiques} & 0,33 & 0,0934 & 0,4928 & 0,7506 & 0,4167 \\
				\textit{atrophie musculaire progressive} & 0,40 & 0,1118 & 0.3489 & 0,8053 & 0,4165 \\
				\textit{ataxie locomotrice progressive} & 0,32 & 0,0386 & 0,4877 &  0,7431 & 0,3974 \\
				\textit{localisations cérébrales} & 0,43 & 0,034 & 0,3017 & 0,8090 & 0,3937 \\
				\textit{hystérie} & 0,2724 & 0,3711 & 0,0442 & 0,8018 & 0,3723 \\
				\textit{athétose} & NA & 0,2029 & 0,274 & 0,8068 & 0,3209 \\
				\textit{maladie de Parkinson} & 0,05 & 0,0775 & 0,333 & 0,7936 & 0,3135 \\
				\textit{aphasie} & 0,0587 & 0,2245 & 0,1334 & 0,7960 & 0,3031 \\
				\textit{trépidation épileptoïde du pied} & 0,0198 & 0,1227 & 0,2919 & 0,7597 & 0,2985 \\
				\textit{astasie-abasie} & NA & 0,0478 & 0,3565 & 0,7375 & 0,2855 \\
				\textit{nystagmus} & 0,0243 & 0,1326 & 0,146 & 0,7474 & 0,2626 \\
				\textit{chorées} & NA & 0,1336 & 0,0701 & 0,8047 & 0,2521 \\
				\textit{épilepsie} & NA & 0,164 & 0,0247 & 0,8199 & 0,2521 \\
				\textit{sclérose en plaques disséminées} & NA  & 0,178 & 0,8089 & NA & 0,2467 \\
				\textit{tremblement} & NA & 0,1686 & 0,0362 & 0,7683 & 0,2432 \\
				\textit{systématisation de l'organisation de la moëlle épinière} & NA & NA & NA & 0,7937 & 0,1888 \\
				\textit{sclérose latérale amyotrophique} & NA & 0,044 & 0,6586 & NA & 0,1757 \\
				\hline
			\end{tabular}
		}
		\caption{Les scores de pertinence pour les termes de référence à partir du corpus \og{}Autres\fg{}.}
	\end{table}
	{\small Moyenne pour tous les termes combinés : 0,3214}
\end{frame}





\begin{frame}{Domaine impactant (\textit{PatternRank}) : \textbf{syndrome de Tourette}}
	
	\begin{table}[h]
		\resizebox{\textwidth}{!}{%
			\centering
			\begin{tabular}{|l|r|r|r|r|r|}
				\hline
				\multicolumn{1}{|c|}{\textbf{Terme}} &
				\multicolumn{1}{c|}{\textbf{TF-IDF} (\textit{TermSuite})} & 	
				\multicolumn{1}{c|}{\textbf{TF-IDF}} & 	
				\multicolumn{1}{c|}{\textbf{BM25}} & 	
				\multicolumn{1}{c|}{\textit{\textbf{PatternRank}}} & 
				\multicolumn{1}{c|}{\textbf{Moyenne}} \\
				\hline
				\rowcolor{yellow!30}\textit{tics convulsifs} & NA & 0,1293 & 0,8385 & 0,8331 & \bolder{0,4502} \\
%				\footnote{Le score le plus élevé parmi les termes \textbf{inventés} par Charcot $\neq$ \textit{hypnose}.} \\
				\textit{épilepsie} & NA & 0,164 & 0,0247 & 0,8199 & 0,2521 \\
				\textit{localisations cérébrales} & 0,43 & 0,034 & 0,3017 & 0,8090 & 0,3937 \\
				\textit{athétose} & NA & 0,2029 & 0,274 & 0,8068 & 0,3209 \\
				\textit{atrophie musculaire progressive} & 0,40 & 0,1118 & 0.3489 & 0,8053 & 0,4165 \\
				\textit{chorées} & NA & 0,1336 & 0,0701 & 0,8047 & 0,2521 \\
				\textit{hystérie} & 0,2724 & 0,3711 & 0,0442 & 0,8018 & 0,3723 \\
				\textit{aphasie} & 0,0587 & 0,2245 & 0,1334 & 0,7960 & 0,3031 \\
				\textit{systématisation de l'organisation de la moëlle épinière} & NA & NA & NA & 0,7937 & 0,1888 \\
				\textit{maladie de Parkinson} & 0,05 & 0,0775 & 0,333 & 0,7936 & 0,3135 \\
				\textit{hypnose} & 0,3543 & 1 & 0,2922 & 0,7738 & 0,6050 \\
				\textit{tremblement} & NA & 0,1686 & 0,0362 & 0,7683 & 0,2432 \\
				\textit{trépidation épileptoïde du pied} & 0,0198 & 0,1227 & 0,2919 & 0,7597 & 0,2985 \\
				\textit{arthropathies tabétiques} & 0,33 & 0,0934 & 0,4928 & 0,7506 & 0,4167 \\
				\textit{nystagmus} & 0,0243 & 0,1326 & 0,146 & 0,7474 & 0,2626 \\
				\textit{ataxie locomotrice progressive} & 0,32 & 0,0386 & 0,4877 & 0,7431 & 0,3974 \\
				\textit{astasie-abasie} & NA & 0,0478 & 0,3565 & 0,7375 & 0,2855 \\
				\textit{sclérose en plaques disséminées} & NA  & 0,178 & 0,8089 & NA & 0,2467 \\
				\textit{embarras parole} & NA & NA & 0,0018 & NA & 0,2341 \\
				\textit{sclérose latérale amyotrophique} & NA & 0,044 & 0,6586 & NA & 0,1757 \\
				\hline
			\end{tabular}
		}
		\caption{Les scores de pertinence pour les termes de référence à partir du corpus \og{}Autres\fg{}.}
	\end{table}
	{\small Moyenne pour tous les termes combinés : 0,3214}
\end{frame}


\begin{frame}{Références à Charcot}
	
	\begin{table}
		\centering
		\resizebox{\textwidth}{!}{%
			\begin{tabular}{|l|p{10cm}|}
				\hline
				\textbf{Terme} & \textbf{Contexte} \\
				\hline
				\rowcolor{yellow!30}\textit{maladie de Parkinson} & [$\dots$] \underline{paralysie agitante} que \textbf{Charcot} a eu raison de dénommer \underline{maladie de Parkinson} \\
				\hline
				\textit{ataxie locomotrice progressive} & [$\dots$] arthropathies [$\dots$] de l'\underline{ataxie locomotrice}, [$\dots$] signalées, pour la première fois, par M. \textbf{Charcot}. \\
				& [$\dots$] \underline{ataxie locomotrice progressive}, constatée par douze médecins, parmi lesquels, [$\dots$] MM. \textbf{Charcot} [$\dots$] \\
				\hline
				\textit{arthropathies tabétiques} & [$\dots$] mais personne n'avait encore décrit des cas d'\underline{arthropathie tabétique}, lorsque, en 1808, M. le professeur \textbf{Charcot} publia la première observation d'arthropathie chez un ataxique.\\
				\hline
				\textit{trépidation épileptoïde du pied} & [$\dots$] \underline{trépidation}, qui se propage parfois à tous les \underline{membres}. Ce spasme, [$\dots$] peut entraîner à sa suite des rétractions fibro-tendineuses analogues à celles que \textbf{Charcot} a décrites chez l'homme, [$\dots$] \\
				\hline
				\textit{sclérose en plaques disséminées} & [$\dots$] une combinaison de la \underline{sclérose en plaques}, bien décrite déjà par \textbf{Charcot} et Vulpian [$\dots$] \\
				\hline
			\end{tabular}
		}
		\caption{Concordance des termes médicaux faisant référence à Charcot -- corpus Autres.}
	\end{table}
\end{frame}


\begin{frame}{Références à Charcot}
	
	\begin{table}
		\centering
		\resizebox{\textwidth}{!}{%
			\begin{tabular}{|l|p{10cm}|}
				\hline
				\textbf{Terme} & \textbf{Contexte} \\
				\hline
				\textit{tremblement} & [$\dots$] \textbf{Charcot} présentait, dans son amphithéâtre, pour démontrer les caractères oscillatoires des diverses variétés de \underline{tremblements}. \\
				\hline
				\textit{nystagmus} & SCLÉROSE EN PLAQUES [$\dots$] [$\dots$], \underline{nystagmus}, [$\dots$]. La difficulté de la résoudre est d'autant plus grande qu'à côté des foyers de sclérose en plaques avec tous les caractères histologiques classiques décrits depuis \textbf{Charcot}, il y a des foyers avec une
				destruction plus ou moins complète des cylindraxes [$\dots$]\\
				\hline
				\rowcolor{yellow!30}\textit{embarras parole} & Lorsqu'on se trouve en présence d'un malade ayant de l'\underline{embarras de la parole} [$\dots$] A. La réponse à la première proposition n'est nullement embarrassante, si l'on veut se rappeler ces paroles de M. le professeur \textbf{Charcot} : [$\dots$] \\
				\hline
				\textit{sclérose latérale amyotrophique} & [$\dots$] \underline{sclérose latérale amyotrophique}, maladie découverte par mon illustre maître \textbf{Charcot}. \\
				\hline
				\rowcolor{yellow!30}\textit{tics convulsifs} & désignée par M. \textbf{Charcot} sous le nom de maladie des \underline{tics convulsifs} \\
				\hline
				\textit{atrophie musculaire progressive} & \textbf{Charcot} et Marie ont décrit la \og{}forme particulière d'\underline{atrophie musculaire progressive}\fg{}\\
				\hline
				\textit{aphasie} & Lorsqu'il y a, dit M. le professeur \textbf{Charcot} (1), suppression de la mémoire pour l'articulation des mots, c'est l'\underline{aphasie} motrice d'articulation ou \underline{aphasie} de Broca qui se présente.\\
				\hline
			\end{tabular}
		}
		\caption{Concordance des termes médicaux faisant référence à Charcot -- corpus Autres (suite).}
	\end{table}
\end{frame}

\begin{frame}{Références à Charcot}
	
	\begin{table}
		\centering
		\resizebox{\textwidth}{!}{%
			\begin{tabular}{|l|p{10cm}|}
				\hline
				\textbf{Terme} & \textbf{Contexte} \\
				\hline
				\textit{astasie-abasie} & [$\dots$] forme particulière d'impuissance motrice dont M. P. Blocqe a donné la
				définition suivante \og{}[$\dots$], et qu'il a désigné sous le nom expressif d'\underline{astasie} et d'\underline{abasie}. C'est là un état morbide sur lequel M. le professeur \textbf{Charcot} est	fréquemment revenu dans ses Leçons du mardi [$\dots$]\\
				\hline
				\textit{athétose} & symptôme désigné par M. W. Hammond sous le nom d'\underline{athétose} [$\dots$] M. \textbf{Charcot} a fait remarquer que cette définition était imparfaite pour les motifs suivants : [$\dots$] \\
				\hline
				\textit{chorées} & \underline{Chorée} hystérique ou rhythmique. C'est à M. le professeur \textbf{Charcot} que nous devons une exacte description de cet état pathologique. \\
				\hline
				\textit{hystérie} & C'est encore à \textbf{lui} [Charcot] que nous devons la connaissance de l'\underline{hystérie} traumatique [$\dots$] \\
				\hline
				\rowcolor{yellow!30}\textit{épilepsie} & M. \textbf{Charcot} a décrit avec le plus grand soin l'\underline{épilepsie} partielle d'origine syphilitique [$\dots$] \\
				\hline
				\rowcolor{yellow!30}\textit{hypnose} & Les trois états de l'\underline{hypnose} décrits par M. \textbf{Charcot} sont devenus classiques, [$\dots$] \\
				\hline
				\textit{systématisation de l'organisation de la moëlle épinière} & \underline{systématisation de la moelle}, synthèses [$\dots$]. Mais \textbf{Charcot}, on l'a vu, est, par nature, enclin à la synthèse. \\
				\hline
				\rowcolor{yellow!30}\textit{localisations cérébrales} & Je vous ai montré \textbf{Charcot}, concourant pour
				la plus grosse part, à l'édification de la doctrine des \underline{localisations cérébrales}, qui est devenue quelque chose comme la préface d'une psychologie nouvelle.\\
				\hline
			\end{tabular}
		}
		\caption{Concordance des termes médicaux faisant référence à Charcot -- corpus Autres (fin).}
	\end{table}
\end{frame}

\begin{frame}{Analyse des cooccurrences}
\begin{table}[h]
	\centering
	\resizebox{\textwidth}{!}{%
		\begin{tabular}{rrrrrr}
			\textbf{Terme} & \textbf{Cooccurrent} & \textbf{Fréquence} & \textbf{Co-fréquence} & \textbf{Indice} & \textbf{Distance moyenne} \\
			\hline
			\rowcolor{yellow!30}chorées & Sydenham & 129 & 63 & \bolder{163} & 1,1 \\
			épilepsie & Jackson & 52 & 34 & 78 & 0,2 \\
			embarras parole & Chervin & 41 & 8 & 17 & 4,5\\
			hystérie & Charcot & 2 968 & 52 & 19 & 5,4\\
			trépidation épileptoïde du pied & Babinski $\cdot$ Charcot & 1 134 & 8 & 12 & 4,8 \\
			hypnose & Braid & 567 & 14 & 12 & 4,7 \\
			nystagmus & Barany & 11 & 3 & 7 & 3,3\\
			tics convulsifs & Charcot & 2 968 & 6 & 8 & 5,2 \\
			localisations cérébrales & Charcot & 2 968 & 9 & 7 & 5,3 \\
			arthropathies tabétiques & Charcot & 2 968 & 4 & 5 & 1,5 \\
			athétose & Hammond & 34 & 2 & 4 & 1,5\\
			sclérose latérale amyotrophique & Charcot & 2 968 & 4 & 3 & 3,5 \\
			atrophie musculaire progressive & Charcot & 2 968 & 4 & 3 & 6,0 \\
			sclérose en plaques disséminées & Charcot & 2 968 & 7 & 3 & 6,3 \\
			aphasie & Charcot & 2 968 & 7 & 2 & 4,0\\
			tremblement & Achard & 137 & 3 & 2 &  2,3\\
			ataxie locomotrice progressive  & Charcot & 2 968 & 3 & 2 & 5,3 \\
			maladie de Parkinson  & Charcot & 2 968 & 3 & 2 & 3,3\\
			astasie-abasie & Charcot & 2 968 & 2 & 2 & 1,5 \\
			systématisation $\dots$ moëlle épinière & NA & NA & NA & NA & NA \\
		\end{tabular}
	}
	\caption{Analyse des cooccurrences des termes médicaux à partir du corpus Autres.}
	\label{tab:cooccurrences}
\end{table}

	
\end{frame}
\end{document}