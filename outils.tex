\begin{frame}{Approches comparées}
%	Méthodes classiques \textit{vs.} celles de l'état de l'art.
	\begin{enumerate}
		\item \textcolor{deepblue}{\textbf{\texttt{TermSuite}}} \hfill {\small\citep{cram2016terminology}}
%		 \footnote{\url{https://github.com/ljpetkovic/Charcot_TermSuite}} \citep{cram2016terminology}
		\begin{itemize}
			\item linguistique (règles) + \textsc{TF-IDF} 
		\end{itemize} 
		\item \textcolor{deepblue}{\textbf{TF-IDF, BM25}} \hfill {\small\citep{robertson1976relevance}}
%		\footnote{\url{https://github.com/ljpetkovic/Charcot_circulations}} \citep{robertson1976relevance}  
		\begin{itemize}
			\item statistique, pondération des termes
		\end{itemize}
		\item \textcolor{deepblue}{\textbf{\textit{PatternRank}}} \hfill {\small \citep{schopf2022}}
%		\footnote{\url{https://github.com/ljpetkovic/Seminaire_doctoral_ObTIC_130325/blob/main/0_main.pdf}}
		\begin{itemize}
			\item apprentissage profond
			\item \texttt{keybert} + \texttt{keyphrase-vectorizers}
			\item utilisation des étiquettes POS
		\end{itemize} 
	\end{enumerate}
	
	\begin{block}{\vspace*{-0.6mm}}
		Traitements effectués en local (1,2) et \textit{via} la plateforme MeSU\footnote{\url{https://sacado.sorbonne-universite.fr/fr/plateforme-mesu/}} (3).
		\begin{itemize}
			\item appliqués à tout le corpus
		\end{itemize}
	\end{block}
	
\end{frame}